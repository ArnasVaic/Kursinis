\sectionnonum{Įvadas}

% TODO parašyti įvadą

% tyrimo objektas
% temos aktualumas
% darbo tikslas
% siekiami rezultatai

% tyrimo objektas - yra tiriama matematinis yag reakcijos modelis

% Matematiniai cheminių reakcijų modeliai padeda geriau suprasti veikia 

\ac{yag} kristalai legiruoti su neodimiu arba kitais lantanoidais yra naudojami kaip kietakūnių lazerių aktyviosios terpės dėl savo geidžiamų optinių savybių. Šios medžiagos lazeriai yra dažnai taikomi gamybos ir medicinos srityse \cite{dubeyExperimentalStudyNd2008, valentiUseErYAG2021}. Šiai medžiagai sintezuoti yra žinoma keletas būdų, tačiau kietafazės reakcijos metodas yra lengviausiai pritaikomas pramoninei gamybai \cite{bhattacharyyaMethodsSynthesisY3AI5O122007}. Matematinis šio proceso modelis galėtų padėti suprasti kokią įtaką maišymas turi šiame procese ir kaip galima būtų padidinti proceso efektyvumą pasirenkant optimalų išmaišymo laiką.

% Itrio aliuminio granatas

Šio \textbf{darbo tikslas} yra sukurti kompiuterinį \acs{yag} reakcijos maišymo modelį ir jį ištirti.

Iškelti darbo uždaviniai:

\begin{enumerate}
% \item Atlikti literatūros analizę difuzijos modelių, \acs{yag} sintezės modelių ir baigtinių skirtumų metodų temomis
\item Sukurti kompiuterinį modelį, kuris įgyvendina \acs{yag} reakcijos modelį
\item Patikrinti kompiuterinio modelio rezultatų korektiškumą ir palyginti juos su eksperimentiniais rezultatais
\item Papildyti kompiuterinį modelį su maišymo procesu
\item Ištirti kompiuterinio modelio rezultatus
\end{enumerate}
\sectionnonum{Įvadas}

% TODO parašyti įvadą

% tyrimo objektas - yag reakcija
% temos aktualumas - aktualu nes yag lazeriai placiai naudojami
% darbo tikslas - yra
% siekiami rezultatai - yra

% tyrimo objektas - yra tiriama matematinis yag reakcijos modelis

% Matematiniai cheminių reakcijų modeliai padeda geriau suprasti veikia 

Itrio aliuminio granato \acs{yag} kristalai legiruoti su neodimiu arba kitais lantanoidais yra naudojami kaip kietakūnių lazerių aktyviosios terpės dėl savo geidžiamų optinių savybių. Šios medžiagos lazeriai yra dažnai taikomi gamybos ir medicinos srityse \cite{dubeyExperimentalStudyNd2008, valentiUseErYAG2021}. Šiai medžiagai sintezuoti yra žinoma keletas būdų, tačiau kietafazės reakcijos metodas yra lengviausiai pritaikomas pramoninei gamybai \cite{bhattacharyyaMethodsSynthesisY3AI5O122007}. Praktikoje, \acs{yag} sintezė, kietafazės reakcijos metodu, užtrunka mažiausiai kelias valandas priklausomai nuo temperatūros, kurioje vykdomas atkaitinimo procesas \cite{mackeviciusCloserLookComputer2012}. Dėl šios priežasties kompiuterinis modelis galėtų padėti efektyviau suprasti kokią įtaką maišymas turi šiam procesui ir kaip jį galima būtų paspartinti.

Šio \textbf{darbo tikslas} yra sukurti kompiuterinį \acs{yag} reakcijos maišymo modelį ir jį ištirti.

Iškelti darbo uždaviniai:

\begin{enumerate}
\item Sukurti kompiuterinį \acs{yag} reakcijos modelį
\item Patikrinti kompiuterinio modelio rezultatų korektiškumą ir palyginti juos su eksperimentiniais rezultatais
\item Papildyti kompiuterinį modelį su maišymo procesu
\item Ištirti kompiuterinio modelio rezultatus
\end{enumerate}
\sectionnonum{Įvadas}

Itrio aliuminio granato \acs{yag} kristalai legiruoti su neodimio arba kitų lantanoidų jonais yra naudojami kaip kietakūnių lazerių aktyviosios terpės dėl savo geidžiamų optinių savybių. Šios medžiagos lazeriai yra dažnai taikomi gamybos ir medicinos srityse \cite{dubeyExperimentalStudyNd2008, valentiUseErYAG2021}. Šiai medžiagai sintezuoti yra žinoma keletas būdų, tačiau kietafazės reakcijos metodas yra lengviausiai pritaikomas pramoninei gamybai \cite{bhattacharyyaMethodsSynthesisY3AI5O122007, zhangNovelSynthesisYAG2005}. Praktikoje, \acs{yag} sintezė, kietafazės reakcijos metodu, užtrunka mažiausiai kelias valandas priklausomai nuo temperatūros, kurioje vykdomas atkaitinimo procesas \cite{mackeviciusCloserLookComputer2012}. Yra žinoma, kad chemikai bando spartinti šią reakcija periodiškai išmaišydami reagentus.

Ivanauskas et al \cite{ivanauskasModellingSolidState2005} pasiūlė matematinį kietafazės \acs{yag} reakcijos modelį ir nustatė reakcijos greičio ir difuzijos konstantas prie tam tikrų temperatūrų, tačiau šis modelis nemodeliuoja minėto išmaišymo proceso. Kompiuterinis modelis, apimantis išmaišymo procesą, galėtų padėti efektyviau suprasti kokią įtaką maišymas turi šiam procesui ir jo trukmei. Šiame darbe įgyvendinsime kompiuterinį reakcijos modelį, pateiksime porą skirtingų išmaišymo modelių ir juos integruosime į kompiuterinį modelį. Nagrinėsime modelio teorinį stabilumą ir modelio rezultatų korektiškumą. Pateiksime ir išanalizuosime įvairiai agreguotus modelio rezultatus.

Šio \textbf{darbo tikslas} yra sukurti kompiuterinį \acs{yag} reakcijos maišymo modelį ir jį ištirti.

Iškelti darbo uždaviniai:

\begin{enumerate}
\item Sukurti kompiuterinį \acs{yag} reakcijos modelį
\item Patikrinti kompiuterinio modelio rezultatų korektiškumą
\item Papildyti kompiuterinį modelį su maišymo procesu
\item Ištirti kompiuterinio modelio rezultatus
\end{enumerate}
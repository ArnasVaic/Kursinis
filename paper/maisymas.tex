
\section{Maišymo modeliavimas}
\subsection{Maišymo proceso modelis}

Ši reakcija gali užtrukti kelias valandas, todėl chemikai bando paspartinti šį procesą išmaišydami reagentus proceso metu. Per visą reakcijos procesą, išmaišymas gali būti vykdomas kelis kartus. Išmaišymas vyksta ištraukus reagentus iš krosnies ir todėl aplinkos temperatūra maišymo metu yra daug kartų žemesnė negu reakcijos vykdymo temperatūra. Išmaišymo procesui modeliuoti nėra apsvarstoma temperatūros kaita - daroma prielaida, kad pradėjus mažinti krosnies temperatūrą, medžiagos nustoja reaguoti. Tada išmaišymą galima modeliuoti kaip momentinį procesą, kuris įvyksta tarp kompiuterinio modelio žingsnių. 

Yra žinoma, kad fizinio maišymo metu, dalelės neskyla ir netampa smulkesnės, todėl maišymą modeliuosime kaip reaguojančių dalelių atsitiktinis sukeitimas vietomis joms priskiriant atsitiktinius pasisukimo kampus. Pradinių sąlygų \eqref{intial-cond} atveju, modeliuojame keturių mikrodalelių reakcija, todėl

Fizinis maišymo procesas procesas nėra determinis



Taigi išmaišymą galima apibrėžti kaip funkciją:

\begin{align}
    \mathcal{M}(c_m) = c_m^*
\end{align}


% aprasyti kaip tas maisymas atrodo is tikro ir kaip skiriasi nuo modelio

% matematinis modelis maisymui
% stop salyga

\subsection{Maišymo procesu papildytos programos rezultatų analizė}


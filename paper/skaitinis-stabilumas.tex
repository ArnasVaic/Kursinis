\subsection{Modelio skaitinis stabilumas}

Norint užtikrinti skaitinį programos stabilumą, reikia užtikrinti, kad visais laiko momentais, visuose diskretizuotos erdvės taškuose, visų medžiagų koncentracijos išliktų ne neigiamos. Parodysime, kad tai užtikrinti, užtenka pasirinkti pakankamai mažą laiko žingsnį $\Delta t$. Pirmiausia įvedame porą konstantų:
\begin{align*}
\mu_x = \frac{D\Delta t}{(\Delta x)^2}, \quad
\mu_y = \frac{D\Delta t}{(\Delta y)^2}
\end{align*}

Tada pertvarkome dviejų dimensijų skaitinį modelį \eqref{numerical-eqs} taip, kad kairėsė lygčių pusėse liktų medžiagų koncentracija laiko momentu $n+1$, o dešinėse lygčių pusėse sugrupuojame narius pagal medžiagų koncentracija skirtinguose diskretizuotos erdvės taškuose:

\begin{subequations} \label{eqs:r-coefs}
    \begin{align}
    c^{n+1}_{1,i,j}&=\underbrace{(1-3\Delta tc^{n}_{2,i,j}-2(\mu_x+\mu_y))}_{R_1}c^n_{1,i,j}+\mu_xc^n_{1,i-1,j}+\mu_xc^n_{1,i+1,j}+\mu_yc^n_{1,i,j-1}+\mu_yc^n_{1,i,j+1} \label{r-coefs1}\\
    c^{n+1}_{2,i,j}&=\underbrace{(1-5\Delta tc^{n}_{1,i,j}-2(\mu_x+\mu_y))}_{R_2}c^n_{2,i,j}+\mu_xc^n_{2,i-1,j}+\mu_xc^n_{2,i+1,j}+\mu_yc^n_{2,i,j-1}+\mu_yc^n_{2,i,j+1} \label{r-coefs2}\\
    c^{n+1}_{3,i,j}&=c^n_{3,i,j}+2\Delta tc^{n}_{1,i,j}c^{n}_{2,i,j} \label{r-coefs3}
    \end{align}
\end{subequations}

% \begin{subequations} \label{mu-eqs}
%     \begin{align}
%     c^{n+1}_{1,i,j}&=c^n_{1,i,j}-3\Delta tc^{n}_{1,i,j} c^{n}_{2,i,j}+\mu_x(c^n_{1,i-1,j}-2c^n_{1,i,j}+c^n_{1,i+1,j})+\mu_y(c^n_{1,i,j-1}-2c^n_{1,i,j}+c^n_{1,i,j+1})\\
%     c^{n+1}_{2,i,j}&=c^n_{2,i,j}-5\Delta tc^{n}_{1,i,j} c^{n}_{2,i,j}+\mu_x(c^n_{2,i-1,j}-2c^n_{2,i,j}+c^n_{2,i+1,j})+\mu_y(c^n_{2,i,j-1}-2c^n_{2,i,j}+c^n_{2,i,j+1})\\
%     c^{n+1}_{3,i,j}&=c^n_{3,i,j}+2\Delta tc^{n}_{1,i,j}c^{n}_{2,i,j}
%     \end{align}
% \end{subequations}

Baziniu atveju, kai $n=0$, medžiagų koncentracija visuose taškuose yra ne neigiama, kaip numatyta pradinėje sąlygoje \eqref{intial-cond}. Darome indukcijos hipotezės prielaidą, kad medžiagų koncentracija visuose diskretizuotos erdvės taškuose, laiko momentu $n$ bus ne neigiama:

\begin{align} \label{induction-assumption}
    c^n_{k,i,j} \geqslant 0, \quad k=1,2,3,\quad i=0,1,\dots,N-1,\quad j=0,1,\dots,M-1
\end{align}

Akivaizdu, kad lygtyje \eqref{r-coefs3}, medžiagos koncentracija $c^{n+1}_{3,i,j}$ negali tapti neigiama dėl prielaidos \eqref{induction-assumption} ir fakto, kad $\Delta t>0$. Pirmos medžiagos lygtyje \eqref{r-coefs1}, galima pastebėti, kad dėmenys su medžiagų koncentracijomis iš aplinkinių diskretizuotos erdvės taškų visada bus ne neigiami dėl prielaidos \eqref{induction-assumption} ir fakto, kad $\mu_x>0$ ir $\mu_y>0$:

\begin{align*}
    \mu_xc^n_{1,i-1,j}+\mu_xc^n_{1,i+1,j}+\mu_yc^n_{1,i,j-1}+\mu_yc^n_{1,i,j+1}\geqslant 0
\end{align*}

Taigi $c^{n+1}_{1,i,j}$ ženklą lemia tik koeficientas $R_1$, todėl įvedame ribojimą, kad $R_1\geqslant 0$. Analogiškai, iš antros medžiagos lygties \eqref{r-coefs2} gauname, kad $R_2\geqslant 0$ ir turime neligybių sistemą:

\begin{align}
  \begin{cases}
    (1-3\Delta tc^{n}_{2,i,j}-2(\mu_x+\mu_y))\geqslant 0\\
    (1-5\Delta tc^{n}_{1,i,j}-2(\mu_x+\mu_y))\geqslant 0
  \end{cases}, \quad i=0,1,\dots,N-1, \quad j=0,1,\dots,M-1
\end{align}

Pertvarke neligybes gauname:

\begin{align}
  \begin{cases}
    \Delta t \leqslant (3c^{n}_{2,i,j}+2D((\Delta x)^{-2}+(\Delta y)^{-2}))^{-1}\\
    \Delta t \leqslant (5c^{n}_{1,i,j}+2D((\Delta x)^{-2}+(\Delta y)^{-2}))^{-1}
  \end{cases}
\end{align}

% \begin{cases}
%     1-\Delta t\left(3c^{n}_{2,i,j}+2D\left(\frac{1}{(\Delta x)^2}+\frac{1}{(\Delta y)^2}\right)\right)\geq 0\implies
%     \Delta t\leq\frac{1}{3c^{n}_{2,i,j}+2D\left((\Delta x)^{-2}+(\Delta y)^{-2}\right)}\\
%     1-\Delta t\left(5c^{n}_{1,i,j}+2D\left(\frac{1}{(\Delta x)^2}+\frac{1}{(\Delta y)^2}\right)\right)\geq 0\implies
%     \Delta t\leq\frac{1}{5c^{n}_{1,i,j}+2D\left((\Delta x)^{-2}+(\Delta y)^{-2}\right)}
% \end{cases}

\newpage
Galima panaikinti laiko žingsnio $\Delta t$ priklausomybę nuo einamojo diskrečios erdvės taško padarius pastebėjimą,
kad laiko žingsnis su didžiausiomis medžiagų kiekių $c^n_{1,i,j}$ bei $c^n_{2,i,j}$ reikšmėmis užtikrintų stabilumą visiem
likusiems diskrečios erdvės taškams, taigi:

\begin{align*}
    \Delta t\leq\min\left(
    \frac{1}{3\max\limits_{(i,j,n)\in[0,N)\times[0,M)\times[0,T)}c^{n}_{2,i,j}
    +2D\left((\Delta x)^{-2}+(\Delta y)^{-2}\right)},\right.\\
    \left. \frac{1}{5\max\limits_{(i,j,n)\in[0,N)\times[0,M)\times[0,T)}c^{n}_{1,i,j}
    +2D\left((\Delta x)^{-2}+(\Delta y)^{-2}\right)}
    \right)
\end{align*}

Taip pat galime atsikratyti laiko žingsnio $\Delta t$ priklausomybės nuo einamojo laiko momento, nes
pagal duotas kraštines sąlygas į sistemą laikui einant nepatenka joks naujas medžiagų $c_1$ ir $c_2$ kiekis.
Taip pat vykstant medžiagų $c_1$ ir $c_2$ reakcijai, bendri šių medžiagų kiekiai uždaroje sistemoje mažės, todėl:

\begin{align*}
    \Delta t\leq\min\left(
        \frac{1}{3\max\limits_{(i,j)\in[0,N)\times[0,M)}c^{0}_{2,i,j}
        +2D\left((\Delta x)^{-2}+(\Delta y)^{-2}\right)},\right. \\
        \left. \frac{1}{5\max\limits_{(i,j)\in[0,N)\times[0,M)}c^{0}_{1,i,j}
        +2D\left((\Delta x)^{-2}+(\Delta y)^{-2}\right)}
    \right)
\end{align*}

Šiuo atveju, iš pradinių sąlygų $\max\limits_{(i,j)\in[0,N)\times[0,M)}c^{0}_{2,i,j}=\max\limits_{(i,j)\in[0,N)\times[0,M)}c^{0}_{1,i,j}=1$, taigi

\begin{align*}
    \Delta t\leq\frac{1}{5+2D\left((\Delta x)^{-2}+(\Delta y)^{-2}\right)}
\end{align*}
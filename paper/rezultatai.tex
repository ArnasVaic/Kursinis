\sectionnonum{Rezultatai ir išvados}

\subsection*{Rezultatai}

Šiame darbe buvo įgyvendinti šie uždaviniai:

\begin{itemize}
    \item Sukurtas kompiuterinis \acs{yag} reakcijos modelis. 
    \item Teoriškai parodyta skaitinio modelio stabilumo sąlyga
    \item Kompiuterinio modelio rezultatai buvo analizuojami ir buvo užtikrinta, kad modelis veikia korektiškai
    \item Pasiūlyti du maišymo modeliai - atsitiktinis ir \enquote{tobulas}
    \item Išmaišymo modeliai integruoti į kompiuterinį \acs{yag} reakcijos modelį
    \item Atlikta papildyto kompiuterinio modelio rezultatų analizė
\end{itemize}

\subsection*{Išvados}

Iš rezultatų analizės galima daryti šias išvadas:

\begin{itemize}
    \item Atsitiktinio maišymo modelio rezultatai neatitinka realybėje pastebimų rezultatų, kai reakcija modeliuojama mažoje srityje, kurioje susiduria tik 4-ios mikrodalelės. Norint iš šio modelio išgauti tikrovę atitinkančius rezultatus yra būtina modeliuoti didesnę erdvės sritį.

    \item \enquote{Tobulo} išmaišymo modelio rezultatai atitinka realybėje pastebimą reakcijos pagreitėjimą.
    
    \item Modeliuojant didesnę erdvės sritį, \enquote{tobulo} išmaišymo modelio rezultatai kinta gana nežymiai, todėl maišymo modeliavimui užtenka modeliuoti mažą reakcijos erdvės sritį su 4-iom skirtingų medžiagų mikrodalelėmis

\end{itemize}